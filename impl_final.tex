\subsection{Implementation and deployment}
\label{sec:deployment}

We have implemented \name, and deployed on 67 servers in 10 of
\company's geo-distributed DCs. Evaluation in the next section is
based on this deployment. \name is implemented with 3621 lines of
golang code~\cite{golang}.

The duplications of the controller are implemented on three different
geo-located zookeeper servers. The Agent Monitor uses \texttt{HTTP
POST} to send controll messages between the controller and servers.
\name uses \texttt{wget} to make each data transfer, and enforce
bandwidth allocation by setting \texttt{--limit-rate} field in each
data transfer. The agent running in each server uses Linux Traffic
Control (\texttt{tc}) to enforce the limit on the total bandwidth
usage of inter-DC multicast traffic.

\name can be seamlessly integrated with any inter-DC communication
patterns. All the applications need to do is to call the \name APIs
that consist of three steps: (1) provide \name with the source DC,
destination DCs, intermediate servers, and the pointer to the bulk
data; (2) install agents on all intermediate servers; (3) and
finally, set the start time of the data transfers. Then \name will
start the data distribution at the specified time. We speculate that
our implementation should be applicable to other companies' DCs too.

\name has several parameters that are set either by adminitrators of
\company, or empirically by evaluation results. These parameters
include: the bandwidth reserved for latency-sensitive traffic (20\%),
the data block size (2MB), and update cycle length (3 seconds).�

