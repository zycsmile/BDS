\section{Background}
\label{sec:motivation}

\jc{Starting from Section 2, it would be paper outline.
Please think each bullet point as a separate paragraph.}

To motivate the need for an inter-DC multicast overlay, we first
study the real-world inter-DC traffic to characterize the key
characteristics of bulk-data multicast traffic
(\Section\ref{subsec:motivation:multicast-traffic}).
We then use \company's infrastructure as a case study to
investigate the opportunities of exploring overlay paths to improve
the performance of bulk-data multicast
(\Section\ref{subsec:motivation:case-for}).
Finally, we draw lessons from real-world incidents and statistics
of the \company's existing multicast protocol
(\Section\ref{subsec:motivation:baseline}), and summarize
the section with challenges to be addressed in \name
(\Section\ref{subsec:motivation:challenges}).


\subsection{The need for bulk-data multicast}
\label{subsec:motivation:multicast-traffic}

\begin{table}[t]
\begin{center}
%\resizebox{\textwidth/2}{!}{
%\begin{tabular}{p{2cm}<{\centering}|p{2cm}<{\centering}}
\begin{tabular}{| c | c|}
\hline
 \rowcolor[gray]{0.9}
\textbf{App Name} & \textbf{Percentage of Multicast Traffic} \\
\hline
Random link\footnotemark[2] & \fillme\\
\hline
App1 & 0.910 \\% 4648.92 vs 41372.56 in GB
\hline
App2 & 0.892\\% 16766.7 vs 138418.12
\hline
App3 & 0.9815\\% 1297.7 vs 68699.97
\hline
App4 & 0.9818\\% 2792.4 vs 150234.25
\hline
App5 & 0.9809\\% 451.22 vs 23134.21
\hline
App6 & 0.9808\\% 964.62 vs 49327.01
\hline
\end{tabular}
%}
\end{center}
\caption{Share of multicast traffic.}
\label{table:rate}
\end{table}
\footnotetext[2]{This link is randomly selected from inter-DC links, whose traffic is monitored by different application types. All the links carry the similar traffic, so the randomly selected link could exhibit good representatives.}

To check the percentage of multicast traffic, we breakdown all Baidu's total traffic volume into non-multicast traffic and the multicast traffic of each application, and then calculate the share of multicast traffic. Table \ref{table:rate} shows that a considerably large fraction of traffic is multicast traffic, despite the application types. This result highlights the importance of bulk-data transmission optimization.



\begin{itemize}

\item Share of multicast traffic: use a bar chart to show the breakdown of all Baidu's total traffic volume into non-multicast traffic, and the multicast traffic of each application. {\em This should show a large fraction of traffic is multicast, and they are from many different applications.}

\item A CDF of number of destination DCs. {\em This should show that most multicast traffic are destined to almost all DCs.}

\item A CDF of size of multicast data files. {\em This should show that most multicast data are bulk data (not small data), and thus focusing on optimizing bulk data multicast is valuable.}

\end{itemize}

\subsection{A case for inter-DC multicast overlay}
\label{subsec:motivation:case-for}

\begin{itemize}

\item Give an illustrative toy example to compare (1) directly sending data to each destination DC, (2) use chain replication, i.e., build a multicast tree with each DC being a node, (3) an optimal solution.
{\em\bf This example is critical!}

\item Briefly explain the basics of Baidu's inter-DC WAN: topology, \# of servers per DC, some estimates on how many disjoint paths are available between two DCs.
{\em The point is that each DC has multiple disjoint paths to fetch data, despite a seemingly tree-like topology.}

\item Show a CDF of $\frac{X_i\rightarrow B}{Y_i\rightarrow B}$, where $X_i\rightarrow B$, $Y_j\rightarrow B$ are the bandwidth between some server in $X$ and $B$, and between some server in $Y$ and $B$, respectively. Assume $X$ needs to go through $Y$ to get to $B$.
{\em The point is that in a substantial fraction of cases, $\frac{X_i\rightarrow B}{Y_i\rightarrow B}>1$, meaning selecting the sender is not straightforward, we do need to consider large decision space.}

\end{itemize}

\subsection{Lessons from a baseline solution}
\label{subsec:motivation:baseline}
\begin{itemize}

\item Briefly describe how \company does multicast today: how to data is forwarded through an intermediate DC? what's the protocol (a receiver-driven decentralized protocol)?
We should also stress that this solution has been running for \fillme years and has been continuously improved over time.

\item Briefly mention other solutions (layered structure, hybrid approach, and why not optimizing pair-wise DC link is not sufficient)

\item Lesson \#1: Completion times are often suboptimal.
Ideally, we need to show a CDF with two lines that one for flow completion time under the optimal overlay decisions, and other showing the completion time of the existing protocol.

\item Lesson \#2: Latency-sensitive online traffic can be impacted (even with QoS enabled at gateway routers)
(1) Use a figure to show that bulk data transfer can cause significant delay on latency-sensitive traffic, and (2) put some concrete numbers to show such delay can cause significant revenue loss.

\end{itemize}

\subsection{Challenges}
\label{subsec:motivation:challenges}

\begin{itemize}

\item Need for centralized decision-making based on global view

\item Need for clean traffic separation.

\end{itemize}


