\section{Appendix}

Suppose we want to send $N$ data blocks to $m$ destination DCs. Without
loss of generality, we consider two cases:
\begin{packeditemize}
\item {\bf A (Balanced):} Each of the $N$ blocks has $k$ duplicas;
\item {\bf B (Imbalanced):} Half blocks have $k_1$ duplicas each, and the
other half have $k_2$ duplicas each, and $k_1<k_2, (k_1+k_2)/2=k$.
\end{packeditemize}
Note that $m>k$, since otherwise, the multicast is already complete.
Next, we prove that in a simplified setting, \name's completion time in A
is strictly less than B.

To simplify the calculation of \name's completion time, we now make a few
assumptions (which are not critical to our conclusion): (1) all servers
have the same upload (resp. download) bandwidth $R_{up}$ (resp. $R_{down}$),
(2) no two duplicas share the same source (resp. destination) server, so
the upload (resp. download) bandwidth of each block is $R_{up}$ (resp.
$R_{down}$).
Now we can write the completion time in the two cases as following:
\begin{equation}
\label{equa:caseK}
\begin{split}
& t_A = \frac{V}{min\{c(l),\frac{k R_{up}}{m-k},
\frac{k R_{down}}{m-k}\}}\\
& t_B = \frac{V}{min\{c(l),
\frac{k_1 R_{up}}{m-k_1},
\frac{k_2 R_{up}}{m-k_2},
\frac{k_1 R_{down}}{m-k_1},
\frac{k_2 R_{down}}{m-k_2}\}}\\
\end{split}
\end{equation}
where $V$ denotes the total size of the untransmitted blocks,
$V=N(m-k)\rho(b)=
\frac{N}{2}(m-k_1)\rho(b)+\frac{N}{2}(m-k_2) \rho(b)$.
In the production system of \company, the inter-DC link capacity $c(l)$ is
several orders of magnitudes higher than upload/download capacity of a
single server, so we can safely exclude $c(l)$ from the denominator in the
equations.
Finally, if we denote $min\{R_{up},R_{down}\}=R$, then
$t_A=\frac{(m-k)V}{k R}$ and
$t_B=\frac{(m-k_1)V}{k_1 R}$.

We can show that $\frac{(m-k)V}{k R}$ is a monotonically decreasing
function of $k$:
\begin{equation}
\label{calculus}
\begin{split}
\frac{d}{dk}\frac{(m-k)V}{k R}
=\frac{d}{dk}\frac{(m-k)^2N\rho(b)}{k R}
=\frac{N \rho(b)}{R}  (1-\frac{m^2}{k^2})<0
\end{split}
\end{equation}
Now, since $k>k_1$, we have $t_A<t_B$.

