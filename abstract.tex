\begin{abstract}

Distributing bulk data across datacenters (DCs) in a timely and
cost-effective manner is critical to large-scale online service
providers.
While recent research has significantly improved the WAN performance
between DCs, we argue that a multicast overlay network that optimally
schedules and delivers data in a way that fully exploits available
overlay paths is essential to achieving desirable performance.
Drawing on the experience of a large online service providers, we
observe two requirements of an inter-DC multicast overlay network:
(1) overlay routing and scheduling needs to be driven by an
up-to-date global view of data delivery status at all servers, and
(2) it must prevent delay of latency-sensitive
traffic caused by bulk data transfers.

This paper presents \name, a near-optimal inter-DC multicast overlay
network that distributing bulk data.
At the core of \name are two design choices.
First, decision making in \name is fully centralized; the \name
controller directly orchestrates servers to split, reorder, and
deliver data dynamically along overlay paths, in order to
circumvent inter-/intra-DC bottlenecks.
Second, \name enforces a dynamic separation of bandwidth allocated
for bulk data transfers and latency-sensitive traffic.
While both design choices introduce costs in performance (i.e.,
\name is unable to update decisions in real time or achieve full
link utilization), our design philosophy is that these costs are
outweighed by the benefits of centralized optimization driven
by a global view.
Using a pilot deployment of \name in one of the largest search
engine service providers, we show that \name achieves 3-5$\times$
speedup over the provider's existing systems and the techniques
widely used in today's content delivery networks.


%Drawing on lessons from the decade-long evolution of similar systems in Baidu,
%\name take a centralized approach, which uses a controller to maintains the
%data delivery status in all servers, and update the scheduling and overlay routing
%decisions in near real time by solving a multicommodity flow problem using a
%near-optimal yet efficient algorithm.
%The key insight underlying \name's centralized architecture is the observation that
%the cost of not being able to adapt to network conditions in real time (such as in
%a decentralized system) is greatly outweighed by the benefits of centralized
%control to optimally update overlay routing and scheduling even at a coarse
%timescale (every few seconds), as well as the resulting system that has less
%complexity.

\end{abstract}
