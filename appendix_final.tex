\section{Appendix}
%\subsection{Optimal result is yielded from the balance availability of blocks}
%\subsection{Proof 1: The optimal result yields from balance availability of blocks}\label{appendix:balance}
Assume there are $m$ destination DCs, and the bulk data is split into $N$ blocks.
%Different scheduling strategies output different block subsets and lead to different intermediate transmission states, finally resulting in different completion time.
Take two intermediate states as examples: {\em A}) Balanced situation. All of the $N$ blocks are balanced duplicated with $k$ duplicates; {\em B}) Imbalanced situation. $\frac{N}{2}$ blocks have $k_1$ duplicates and $\frac{N}{2}$ blocks have $k_2$ duplicates, $k_1+k_2=2k$ and $k_1<k<k_2$. Let $t_A$ denote the completion time of case {\em A} and $t_B$ denote that of case {\em B}, and we then compare $t_A$ and $t_B$.
%First we classify the $m$ DCs into two sets:
%
%$N_b = \{DC_i|DC_i \text{ has already downloaded block $b$}\}$
%
%$N'_b = \{DC_i|DC_i \text{ hasn't downloaded block $b$}\}$

Here we list some specific assumptions, although they are real cases in \company. First, different blocks are duplicated in different servers, so that the total upload bandwidth can be the sum of all $R_{up}$ of source servers. Second, the bottlenecks of the two cases are the same. Third, link capacity is much larger than the server rate limit. Therefore, the expected transmission time can be calculated by Equation \ref{equa:caseK}:
 
%where $|N_b| + |N'_b| = m$.
%For the blocks with $k$ duplicates, the remaining $m-k$ transmissions have $k$ optional data sources,
%Let $R_{up}$/$R_{down}$ denote the upload/download rate limits of each server,
%and the

\begin{equation}
\label{equa:caseK}
\begin{split}
& t_A = \frac{\mathbb{S}}{min\{c(l),\frac{R_{up}(s)\times k}{m-k},\frac{R_{down}(s')\times k}{m-k}\}}\\
& t_B = \frac{\mathbb{S}}{min\{c(l),\frac{R_{up}(s)\times k_1}{m-k_1},\frac{R_{up}(s)\times k2}{m-k_2},\frac{R_{down}(s')\times k_1}{m-k_1},\frac{R_{down}(s')\times k_2}{m-k_2}\}}\\
\end{split}
\end{equation}

%\begin{equation}
%\label{caseK}
%t_2 = \frac{N/2\times(m-k_1)\times \rho_b + N/2\times(m-k_2)\times \rho_b} {min\{c(l),\frac{R_{up}(s)\times (k_1+k_2)}{(m-k_1)+(m-k_2)},R_{down}(s')\}}
%\end{equation}

where $\mathbb{S}=N\cdot (m-k)\cdot \rho(b)=\frac{N}{2}\cdot (m-k_1)\cdot \rho(b) + \frac{N}{2}\cdot (m-k_2)\cdot \rho(b)$, denoting the size of the remaining blocks, and $\frac{R_{up}(s)\times k}{m-k}$ is the average upload bandwidth for a $k$-duplicated transmission. Let $R=min\{R_{up}(s),R_{down}(s')\}$, then $t_A$ and $t_B$ can be simplified as follows:

\begin{equation}
\label{equa:simp}
t_A = \frac{\mathbb{S}}{\frac{R\times k}{m-k}}, t_B = \frac{\mathbb{S}}{\frac{R\times k_1}{m-k_1}}\\
\end{equation}

Then the completion time is a monotonically decreasing function of $k$ according to the following equation:
\begin{equation}
\label{calculus}
\begin{split}
t'(k) &= \frac{d(t(k))}{d(k)} = \frac{d(\frac{\mathbb{S}\times (m-k)}{R\times k})}{d(k)} \\
    &= \frac{d(\frac{N\times (m-k)^2\times \rho(b)}{R\times k})}{d(k)} = \frac{N\times \rho(b)}{R} \times (1-\frac{m^2}{k^2})
\end{split}
\end{equation}

%To compare $t_1$ and $t_2$, there are two situations:
%
%\begin{itemize}
%\item when $min\{c(l),\frac{R_{up}(s)\times k}{m-k},R_{down}(s')\} = c(l)$ or $R_{down}(s')$, with $\mathbb{S}_k<\mathbb{S}_{k_1}$, so $t_1<t_2$
%
%\item when $min\{c(l),\frac{R_{up}(s)\times k}{m-k},R_{down}(s')\} = \frac{R_{up}(s)\times k}{m-k}$, then the completion time is a monotonically decreasing function of $k$ according to the following equation:
%
As $k<m$, so $t'(k)<0$, thus, $t_A<t_B$ due to $k>k_1$. Therefore we can conclude that the optimal result is yielded from the balance availability of all data blocks.

%\begin{equation}
%\label{caseK}
%t_1 = \frac{\mathbb{S}}{min\{c(l),\frac{R_{up}(s)\times k}{m-k},R_{down}(s')\}}
%\end{equation}
%where $\mathbb{S}=\displaystyle{\sum_{b \in \mathbb{B}}} (1-w_{b,s}^{(T_k)}) \cdot \rho(b)$ denotes the size of the remaining data, and $c(l) > R_{up}/m$ (server upload bandwidth is the bottleneck). This equation is a monotonically decreasing function of $k$ according to the following calculus.
%
%\begin{equation}
%\label{calculus}
%\begin{split}
%t'(k) &= \frac{d(t(k))}{d(k)} = \frac{d(\frac{\mathbb{S}\times (m-k)}{R_{up}\times k})}{d(k)} \\
%    &= \frac{d(\frac{m\times \mathbb{S}}{R_{up}\times k})}{d(k)} = -\frac{m\mathbb{S}}{R_{up}\times k^2} < 0
%\end{split}
%\end{equation}
%
%Therefore, in case 2, for those blocks with $k1$ ($k1<k$) duplicates, the completion time will be larger than $t_1$, i.e., $t_2 \geq t_{k1} > t_1$. Therefore, we can conclude that the optimal result is yielded from the balance availability of all data blocks.

%\subsection{Proof 2: Near Optimality of \name}\label{appendix:optimality}
%
%Here we present the proof of rounding the integer solution.
%
%\name distributes each task in an optimal manner given by the maximum concurrent flow (MCF) solution \cite{reed2012traffic}. However, the MCF solution is a linear relaxation of the original integer problem and in practice the solution requires that flows are routed as integers of the block size flow. There are a number of strategies using the output of the MCF for the integer path allocation, however, none of them will be strictly optimal as the original problem is NP-complete. A common strategy is to use \emph{randomized-rounding}. Randomized-rounding was first analyzed in detail for some relaxation approaches by Raghavan and Thompson \cite{Raghavan1987} and a similar approach is followed here for the specific case of the MCF relaxation for which they do not provide a solution. A flow through an arbitrary edge \(e\) is denoted as \(f_e=\sum_{e \in p_\lambda} F_{B_{i,j},p_\lambda} \; \forall \; p_\lambda\) where $F^*$ is the optimum value in the MCF problem that is equivalent to $f^*$ in the original problem. $f_e$ is the optimal MCF flow through edge $e$ that in the integer solution is approximated from the flow of multiple blocks each of flow rate $b$. To perform the randomized rounding, first, each of the \emph{fractional flows}  $F_{B_{i,j},p_\lambda}$ is truncated to an \emph{integer flow} that is constructed from a number of\emph{ block flows}. Each integer flow is denoted as $\left \lfloor F_{B_{i,j},p_\lambda} \right \rfloor = kb \leq F_{B_{i,j},p_\lambda}$ where $k$ is strictly the largest positive integer, or zero, to meet the constraint. Then each integer flow is either kept at $kb$ or \emph{rounded} up with probability $(F_{B_{i,j},p_\lambda} - \left \lfloor{F_{B_{i,j},p_\lambda}} \right \rfloor)/b$ to $(k+1)b$. This rounding probability means that the expected value of the integer flows is $f_e$, the solution of the optimal linear MFC. The problem is then to determine the likelihood that this rounding will cause the capacity constraint on an edge $e$ to be exceeded. To aid the discussion, a \emph{residual flow}  $f^\prime_e=f_e - \sum_{e \in p_\lambda} \left \lfloor{F_{B_{i,j},p_\lambda}} \right \rfloor $ is used to describe the flow that needs to be added to the truncated flows to obtain the optimal solution obtained from MFC.
%
%As each flow rounding is an independent random variable, we can determine the likelihood of exceeding the capacity constraint by applying the following well-known Chernoff bound~\cite{chernoff1952}:
%\begin{equation}
%  \label{chernoff}
%  \Pr(X > (1+\delta)\mu) \leq e^{\frac{2\delta^2\mu^2}{n(b-a)^2}}
%\end{equation}
%where $X=X_1 + \ldots X_n$ is the sum of independent, bounded, random variables $a \leq X_i \leq b$ and $\mu = E{(X)} $. This bound describes the probability that a summation of random variables exceeds the expected value of the sum by a factor of $\delta$. The random variables are the rounded flows which have an expected summation $\mu = f^\prime_e$. Consider an edge, $e$, that has its capacity fully used by the linear MFC solution. Using the Chernoff bound from Equation~\ref{chernoff}, note that $a=0$ and the highest value of the rounded flow added is one block with flow $b$, we find that the integer flow is beyond optimal and the residual flow is bounded by:
%\begin{equation}
%  \label{block}
%  \delta \leq \frac{b}{r} \sqrt{\frac{n}{2}\ln\epsilon}
%\end{equation}
%with probability $1-\epsilon$. It should be noted that as the size of the blocks becomes small compared to the overall flow $f_e$, the probability that the linear flow is exceeded by a factor $\delta$ becomes exponentially small due to the exponential form of the Chernoff bound.
